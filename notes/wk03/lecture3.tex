\documentclass[11pt, notitlepage]{report}

	\usepackage[margin=1in]{geometry}
	\usepackage{amsmath,amsthm,amssymb,amsfonts}
	\usepackage{enumitem}
	
	\newcommand{\N}{\mathbb{N}}
	\newcommand{\Z}{\mathbb{Z}}
	\newcommand{\R}{\mathbb{R}}
	\newcommand{\A}{\alpha}
	\newcommand{\ora}[1]{\overrightarrow{#1}}
	\usepackage[parfill]{parskip}
	\usepackage{mathtools}
	\newenvironment{solution}{\paragraph{\small Solution:}}{\hfill}
	\newenvironment{theorem}{\paragraph{Theorem:}}{\hfill}
	\newenvironment{definition}{\paragraph{Definition:}}{\hfill}
	\newenvironment{problem}[2][Problem]{\begin{trivlist}
	\item[\hskip \labelsep {\bfseries #1}\hskip \labelsep {\bfseries #2.}]}{\end{trivlist}}
	
	\usepackage{pgfplots}
	\usetikzlibrary{arrows}
	\usetikzlibrary{decorations.markings}
	\usetikzlibrary{datavisualization}
	\usetikzlibrary{datavisualization.formats.functions}
	%\usepackage{pstricks-add}

	\pgfplotsset{every axis/.append style={
	                   axis x line=middle,    % put the x axis in the middle
	                   axis y line=middle,    % put the y axis in the middle
	                   axis line style={<->,color=gray}, % arrows on the axis
	                   xlabel={$x$},          % default put x on x-axis
	                   ylabel={$y$},          % default put y on y-axis
	           }}
	\pgfplotsset{compat=1.15}
	
	\newcommand{\pgraph}[4]{
		\begin{center}
		
		\begin{tikzpicture}
		\begin{axis}[
		   trig format plots=rad,
		   axis equal,
		   grid=both
		]
		\addplot [domain=#3:#4, variable=\t, samples=50, black, decoration={
		   markings,
		   mark=between positions 0.2 and 1 step 4em with {\arrow [scale=1.5]{stealth}}
		   }, postaction=decorate]
		({#1}, {#2});
		
		\end{axis}
		\end{tikzpicture}
		
		\end{center}
	}


   \newcommand{\cgraph}[3]{
	   \begin{center}
	
	   \begin{tikzpicture}
	   \begin{axis}[
	       trig format plots=rad,
	       axis equal,
	       grid=both
	   ]
	   \addplot [domain=#2:#3, variable=\x, samples=50, black, decoration={
	       markings,
	       mark=between positions 0.2 and 1 step 4em with {\arrow [scale=1.5]{stealth}}
	       }, postaction=decorate]
	   {#1};
	
	   \end{axis}
	   \end{tikzpicture}
	
	   \end{center}
	}


	
	\makeatletter
	\newcommand*{\toccontents}{\@starttoc{toc}}
	\makeatother


\begin{document}
   \title{Physics 5a: Homework 1}
   \author{Abhijay Bhatnagar}
   \maketitle
   \toccontents



\setcounter{secnumdepth}{0} %% no numbering
\section{Newtons Laws}

K.K. Ch. 2

\subsection{Known properties}

We know:

Objects have \textbf{mass}

Four fundamental forces:
\textbf{Gravity, Electromagnetic Force, Strong Force, Weak Force}

\subsection{First Law}

In English: An object at rest stays at rest, an object in motion stays in motion.

More mathematical rigorous: If $\ora{F}_i=0$, then $\ora{v}_i$ is constant $(\ora{a}_i=0)$. (True only in an "inertial frame")


\subsubsection{Inertial Frames}

Graph outlining R(t) to a point that extends another vector r-prime, which has the combined vector of r.

R is location of plane.
\begin{align*}
\ora{r}(t) &= \ora{R}(t)+\ora{r'}(t)  \\
\ora{v}(t) &= d/dt\ora{R}(t)+d/dt\ora{r'}(t) = V(t) + v'(t) \\
\ora{a}(t) &= d/dt\ora{V}(t)+\ora{a'}(t) 
\end{align*}

Case 1) $R(t) = R_0+tV => a(t) = a'(t)$

Case 2) $R(t) = R_0 + 1/2t^2 \ora{A}
=> a(t) \neq a'(t)$

In english: People can disagree whether or not things are moving, i.e. frames of reference, and a frame of reference is inertial if it doesn't start accelerating in the frame without forces from the frame.

\subsection{Second Law}

$\ora{F}_i=m_i\ora{a}_i$

\[\ora{F}_i= F_{2->1} (r_1,r_2,v_1,v_2) + F_{3->1} (r_1,r_3,v_1,v_3)
= \sum_{i}{F_{i->1}} \]


\subsubsection{Example based elaborations}

Gravity: \begin{align*}
	F_{E->1}=-m_ig\hat{j}
\end{align*}

Diagram of tree and earth. Force $a=F_{earth -> apple}$ pulling down on  the apple, a little algebra later you have $a=-g\hat{j}$.\\

Contact force: 

Ball on slope diagram. Force is pushing on ball, then another diagram showing a ball on a vertical wall showing how the contact isn't pushing in that case. The contact force of $\ora{F}_{1->2} = \hat{n} \cdot f$. 

More examples of normal force.

\subsubsection{Tension}

You can only ever pull with a rope. Tension can be worked out with normal forces and seeing what things aren't falling, i.e. if gravity is pulling down on a hanging block, the rope must be pulling up.

$F_{s->1}=T\hat{t}$.

\subsection{Third Law}
\subsection{Fourth Law}


\end{document}
