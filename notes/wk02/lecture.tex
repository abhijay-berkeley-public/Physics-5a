\documentclass[11pt, notitlepage]{report}

   \usepackage[margin=1in]{geometry}
   \usepackage{amsmath,amsthm,amssymb,amsfonts}

   \newcommand{\N}{\mathbb{N}}
   \newcommand{\Z}{\mathbb{Z}}
   \newcommand{\R}{\mathbb{R}}
   \newcommand{\A}{\alpha}
   \usepackage[parfill]{parskip}
	\newenvironment{solution}{\paragraph{\small Solution:}}{\hfill}
	\newenvironment{theorem}{\paragraph{Theorem:}}{\hfill}
	\newenvironment{definition}{\paragraph{Definition:}}{\hfill}

   \usepackage{pgfplots}
   \usetikzlibrary{arrows}
   \usetikzlibrary{decorations.markings}
   \usetikzlibrary{datavisualization}
   \usetikzlibrary{datavisualization.formats.functions}
   %\usepackage{pstricks-add}

	\pgfplotsset{every axis/.append style={
	                   axis x line=middle,    % put the x axis in the middle
	                   axis y line=middle,    % put the y axis in the middle
	                   axis line style={<->,color=gray}, % arrows on the axis
	                   xlabel={$x$},          % default put x on x-axis
	                   ylabel={$y$},          % default put y on y-axis
	           }}
	\pgfplotsset{compat=1.15}
   
	\newcommand{\pgraph}[4]{
		\begin{center}
		
		\begin{tikzpicture}
		\begin{axis}[
		   trig format plots=rad,
		   axis equal,
		   grid=both
		]
		\addplot [domain=#3:#4, variable=\t, samples=50, black, decoration={
		   markings,
		   mark=between positions 0.2 and 1 step 4em with {\arrow [scale=1.5]{stealth}}
		   }, postaction=decorate]
		({#1}, {#2});
		
		\end{axis}
		\end{tikzpicture}
		
		\end{center}
	}


   \newcommand{\cgraph}[3]{
	   \begin{center}
	
	   \begin{tikzpicture}
	   \begin{axis}[
	       trig format plots=rad,
	       axis equal,
	       grid=both
	   ]
	   \addplot [domain=#2:#3, variable=\x, samples=50, black, decoration={
	       markings,
	       mark=between positions 0.2 and 1 step 4em with {\arrow [scale=1.5]{stealth}}
	       }, postaction=decorate]
	   {#1};
	
	   \end{axis}
	   \end{tikzpicture}
	
	   \end{center}
	}


	\newenvironment{problem}[2][Problem]{\begin{trivlist}
	\item[\hskip \labelsep {\bfseries #1}\hskip \labelsep {\bfseries #2.}]}{\end{trivlist}}
	
	\makeatletter
	\newcommand*{\toccontents}{\@starttoc{toc}}
	\makeatother


\begin{document}
   \title{Math 54: Week 2 Discussion}
   \author{Abhijay Bhatnagar}
   \maketitle
   \toccontents



\setcounter{secnumdepth}{0} %% no numbering
\section{Review From Last Week}

Configuration $\{q_i\} \implies$ picture of system (no $v$.) \\
Trajectory $\{q_i(t)\} \implies$ movie of system: $v=\frac{dq}{dt}, a= \frac{d^2q}{dt^2}$ \\
Newton $\{{q_i(t)\, v_i(0)} \implies$ $\{q_i(t)\}$ (Phase space, config not enough)

\section{Kinematics}

\subsection{Taylor series expansion for velocity..?}

\begin{align}
	v(t) &= -\frac{g}{a}(1-e^(\alpha t)) \\
	v(0) &= 0
\end{align}

T.E. $= e^(-\alpha t)$

Use Taylor series approximation for T.E. 
\begin{align}
	v(t) &= -\frac{g}{a}(1-(1-\A t + 1/2 \A^2t^2+...)) \\
		 &= -gt + \frac{g}{2}\A t
\end{align}

The taylor expansion is good when $x$ is small $\implies \A t << 1 \implies t << \frac{1}{\A}$.

\subsection{Visualizations of positional derivatives}

$x(t) =$ \cgraph{-(x-2)^2+2}{2-sqrt(2)}{2+sqrt(2)}
$v(t) =$ \cgraph{-2*x}{2-sqrt(2)}{2+sqrt(2)}
$a(t) =$ \cgraph{-2}{2-sqrt(2)}{2+sqrt(2)}

\newpage

\subsection{Derivative exercises}

\begin{enumerate}
	\item Exercise in finding 1st and 2nd derivatives of $x(t) = cost(\A t)$
	\item Exercise in finding 1st and 2nd derivatives of $x(t) = \frac{1}{2} at^2$
	\item Exercise in finding 1st and 2nd derivatives of $x(t) = \frac{1}{2} at^2 + v_0 t + x_0$

\end{enumerate}

\subsection{Integral exercise}

\begin{enumerate}
	\item Exercise in finding 1st and 2nd antiderivatives of $x(t) = x_0$

\end{enumerate}

\subsection{Considering Dimensions}

\begin{align}
	\overrightarrow{r}(t) &= (x(t), y(t)) \\
						  &= x(t)\hat{i} + y(t)\hat{j}
\end{align}

When converting from different coordinate perspectives:


\begin{align}
	\overrightarrow{r}(t) &= x(t)\hat{i} + y(t)\hat{j} = \tilde{x}\tilde{\hat{i}} + \tilde{y}\tilde{\hat{j}} \\
	x &\neq \tilde{x}\\
	y &\neq \tilde{y} \\
	\text{... get from notes}
\end{align}

\subsection{Vectors}

\begin{enumerate}
	\item Adding $\overrightarrow{v}+\overrightarrow{u}$
	\item Length $\|\overrightarrow{v}\|$
	\item Dot $\overrightarrow{v}\cdot\overrightarrow{u}$
	\item Cross $\overrightarrow{v} \text{x} \overrightarrow{u}$
\end{enumerate}


\newpage

\subsection{Ball-in-cup-esque exercise}

\scalebox{0.5}

Exercise in finding when a ball shot

\cgraph{-0.5*(x-2)^2+1}{2-sqrt(2)}{2+sqrt(2)}

\end{document}
